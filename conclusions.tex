%-----------------------------------------------------------Header Starts--------------------------------------

\documentclass[main.tex]{subfiles}

%-----------------------------------------------------------Header Ends--------------------------------------


\graphicspath{ {./image_intro/} }



\begin{document}

\chapter{Conclusions}

\thispagestyle{empty}

\pagestyle{fancy}
\fancyhf{}

\rhead{Page \thepage}
\lhead{\chaptername \ \thechapter}

The objective of this thesis was to investigate the role of oxygen vacancies on ferromagnetism in TiO$_{2}$ which is one of the most promising oxide dilute magnetic semiconductors. In order to create oxygen vacancies, Ti$^{4+}$ was substituted by Sm$^{3+}$ in TiO$_{2}$ nanoparticles which has $\sim$40$\%$ larger ionic radius (109.8 pm) than Ti$^{4+}$ (74.5 pm) ion. Due to this large variation in ionic radius, significant distortion in TiO$_{2}$ lattice is expected along with the suppression in grain growth and structural phase transition phenomena. Despite of the diffrence in ionic radius which indicates less solid solubility of Sm$^{3+}$ in TiO$_{2}$, high concentration of Sm (20 mol$\%$) was added to investigate the formation of second phase of dopants and their effects in structural, optical and magnetic properties of Anatase TiO$_{2}$ beyond solid solubility limit. The salient features of this thesis may be summarized as follows: \\

\begin{itemize}

	\item Pristine and Sm:TiO$_{2}$ (from 0 to 20 mol$\%$ Sm) nanoparticles were synthesized by solgel method.\\

	\item X-ray diffraction analysis showed that all Bragg peaks observed in the line scans of all samples were completely matched with Anatase phase of TiO$_{2}$. Above 10 mol$\%$ of Sm substitution, a broad hump was detected in X-ray diffraction patterns which was identified as the amorphous Sm$_{2}$O$_{3}$ phase.\\

	\item The substitution of Sm was found to suppress the grain size from 53($\pm10$) nm of pristine Anatase to 10($\pm3$) nm of 20 mol$\%$ Sm substituted TiO$_{2}$.\\

	\item High resolution TEM images and electron diffraction study showed that no metallic clusters of Sm$^{3+}$ ions or crystalline Sm$_{2}$O$_{3}$ were present in the samples within the detection limit.\\

	\item SAED and STEM-EDX analysis posit that the amorphous Sm$_{2}$O$_{3}$ phase might be present as atomically thin layer around the Anatase phase.\\

	 \item Optical property analysis by photoluminescence and UV-Vis-NIR spectroscopy suggest that all samples exhibit indirect bandgap and the Sm incorporation reduced the bandgap from 3.0 eV of pristine Anatase to 2.47 eV of 20 mol$\%$ Sm:TiO$_{2}$ sample.\\

	\item The Sm addition increased the concentration of oxygen vacancy which create shallow trap centers just below the conduction band. The presence of these trap centers manifests the visible photoluminescence due to the recombination of mobile electrons in the trap centers with the holes in the valence band.\\ 

	\item Magnetization vs. applied field characteristics show that Sm:TiO$_{2}$ samples show dilute ferromagnetism at room temperature (300 K). At 5 K temperature, an evolution of paramagnetic behavior along with ferromagnetic response was noticed in the M-H graphs of all samples.\\
	
\end{itemize}
\FloatBarrier



\thispagestyle{fancy}


\end{document}







