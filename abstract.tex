%-----------------------------------------------------------Header Starts--------------------------------------

\documentclass[main.tex]{subfiles}

%-----------------------------------------------------------Header Ends--------------------------------------


\thispagestyle{fancy}
\fancyhf{}
%\rhead{Page \thepage}
%\lhead{\chaptername \ \thechapter}
\cfoot{Page \thepage}


\begin{document}


\chapter*{\vspace{-2cm}Abstracts}

The fascinating concept of substituting the cation of oxide based dilute magnetic semiconductors (DMS) with transition/rare earth metal ions shows tremendous prospects because of their usefulness in ultrafast spin-charge transport phenomena and their applications. The aim of this thesis is focused on studying the effects of $Sm^{3+}$ substitution for $Ti^{4+}$ ion in TiO$_{2}$ lattice from 0 mol$\%$ to as high as 20 mol$\%$. Both X-Ray diffraction and electron diffraction analysis show that the substitution of Sm inhibited the grain growth and phase transition from Anatase to Rutile. The particle size distribution estimated from Transmission Electron Microscopy (TEM) shows that particle size was reduced from 53($\pm$10) nm to 10($\pm$3) nm due to addition of Sm content. The photoluminescence  and UV-Vis-NIR spectroscopy suggest that all samples exhibit indirect bandgap and addition of Sm content reduces the bandgap because of the presence of shallow trap centers created by oxygen vacancies just below the conduction band. The magnetization vs. applied field (M-H) exhibit dilute ferromagnetic behavior at 300 K for all samples while an evolution of paramagnetic response along with ferromagnetic behavior was noticed with increasing Sm content at 5 K which might be attributed to the presence of the amorphous samarium oxide. Such promising results suggest that the role of oxygen vacancies in formation of amorphous second phase of bulk dopants which might contribute to the net ferromagnetic behavior of dilute magnetic semiconductors (DMS) are worthy of further investigation. 



\thispagestyle{fancy}
\newpage

\end{document}







